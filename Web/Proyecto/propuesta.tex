\documentclass[11pt]{scrartcl}

\usepackage[sexy]{evan}
\usepackage{pgfplots}
\pgfplotsset{compat=1.15}
\usepackage{mathrsfs}
\usetikzlibrary{arrows}
\usepackage{graphics}
\usepackage{tikz}
\usepackage{ amssymb }
\usepackage[dvipsnames]{xcolor}
\usepackage[utf8]{inputenc}
\usepackage{longtable}
\usepackage{ragged2e}
\usepackage{listings}


\definecolor{noseve}{RGB}{242,242,242}

\newcommand{\camod}[1]{\frac{\ZZ}{#1 \ZZ}}
\newcommand{\modm}[1]{\text{ mod } #1}
\newcommand{\campm}[1]{\frac{\ZZ}{m\ZZ}}

\usepackage{epigraph}
\renewcommand{\epigraphsize}{\scriptsize}
\renewcommand{\epigraphwidth}{60ex}


\definecolor{dcol0}{HTML}{C8E6C9}
\definecolor{dcol1}{HTML}{D4E9B3}
\definecolor{dcol2}{HTML}{E5ED9A}
\definecolor{dcol3}{HTML}{FFF59D}
\definecolor{dcol4}{HTML}{FFE082}
\definecolor{dcol5}{HTML}{FFCC80}
\definecolor{dcol6}{HTML}{FFAB91}
\definecolor{dcol7}{HTML}{F49890}
\definecolor{dcol8}{HTML}{E57373}
\definecolor{dcol9}{HTML}{D32F2F}

\makeatletter
\newcommand{\getcolorname}[1]{dcol#1}
\makeatother

\newcommand{\dif}[1]{%
    \edef\colorindex{\number\fpeval{floor(#1)}}%
    \edef\fulltext{#1}%
    \colorbox{\getcolorname{\colorindex}}{%
        \ifnum\colorindex>8
            \textbf{\textcolor{white}{\,\fulltext\,}}%
        \else
            \textbf{\textcolor{black}{\,\fulltext\,}}%
        \fi
    }%
}
% Variable para dificultad (inicial 0)
\newcommand{\thmdifficulty}{0}

% Comando para asignar dificultad antes del problema
\newcommand{\problemdiff}[1]{\renewcommand{\thmdifficulty}{#1}}

% Estilo del problema que incluye dificultad antes del título
\declaretheoremstyle[
    headfont=\color{blue!40!black}\normalfont\bfseries,
    headformat={%
      \dif{\thmdifficulty}\quad \NAME~\NUMBER\ifx\relax\EMPTY\relax\else\ \NOTE\fi
    },
    postheadspace=1em,
    spaceabove=8pt,
    spacebelow=8pt,
    bodyfont=\normalfont
]{problemstyle}

    \declaretheorem[style=problemstyle,name=Problema,sibling=theorem]{problema}
    \declaretheorem[style=problemstyle,name=Problema,numbered=no]{problema*}

%\usepackage[
%backend=biber,
%style=alphabetic,
%sorting=ynt
%]{biblatex}
%\addbibresource{referencias.bib}

\newcommand{\indicacion}[1]{\noindent\textit{\small #1}}


\title {Propuestas}
%\subtitle{Sistemas de Medicion y Control 18MPEDS0730 \\ Ago-Dic 2025 \\ Centro de Enseñanza Tecnica Industrial Plantel Colomos\\Tgo. en Desarrollo de Software \\ Academia: Sistemas Electrónicos\\Profesor: Diana Marisol Figueroa Flores }
\date{1 de Septiembre de 2025}
\author{Emmanuel Buenrostro 22300891 7F1 \\ \and Emilio Mateo Rico Garcia 22300895 7F1 \\ \and Emiliano Arzate Gutierrez 22300929 7F1}


\begin{document}

\maketitle
\begin{center}
   \includegraphics[scale=0.15]{../../cetilogo.jpg} 
\end{center}
\newpage

\section{Propuesta 1: Pagina web para visualización de problemas}

Un sitio web donde se puedan visualizar problemas de matemáticas guardados en una base de datos. 

Cada usuario puede agregar problemas, ver la lista de problemas, y para cada problema verlo en una pagina en especifico. 
En cada problema tiene un conjunto de tags, además de un area y una dificultad.





\newpage

\section{Propuesta 2: Manager de Tareas}

Una págins web simple que te permite agregar tareas de distintas materias, para tener una lista organizada de todos tus pendientes. La página siempre te muestra cuantas tareas tienes pendientes para hoy, para esta semana, y en general. Así mismo puedes marcar tareas como completadas, o puedes editar la información de tus tareas. También puedes escoger de que materia es cada tarea. Cuando una tarea se marca como completada, se envía a otra lista mostrando únicamente las tareas completadas. Cada usuario tiene una cuenta en la que puede ver sus tareaas, por lo que es necesario iniciar sesión / crear una cuenta para poder ingresar.

Todos estos problemas se estarán almacenando en una base de datos, así como las cuentas de los usuarios. El plan es utilizar PHP para todo lo referente al CRUD, utilizar CSS para darle un estilo único pero compresnbie, y HTML para poder formar el esqueleto de nuestra plataforma / manager para tareas

    \end{document}