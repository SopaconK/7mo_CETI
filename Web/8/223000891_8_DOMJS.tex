\documentclass[11pt]{scrartcl}

\usepackage[sexy]{evan}
\usepackage{pgfplots}
\pgfplotsset{compat=1.15}
\usepackage{mathrsfs}
\usetikzlibrary{arrows}
\usepackage{graphics}
\usepackage{tikz}
\usepackage{ amssymb }
\usepackage[dvipsnames]{xcolor}
\usepackage[utf8]{inputenc}
\usepackage{longtable}
\usepackage{ragged2e}
\usepackage{listings}
\definecolor{red1}{RGB}{255, 153, 153}
\definecolor{green1}{RGB}{204, 255, 204}
\definecolor{blue1}{RGB}{204, 255, 255}
\definecolor{yellow1}{RGB}{255, 247, 160}

\definecolor{red2}{RGB}{255, 102, 102}
\definecolor{green2}{RGB}{108, 255, 108}
\definecolor{blue2}{RGB}{94, 204, 255}
\definecolor{yellow2}{RGB}{255, 250, 104}

\definecolor{red2.5}{RGB}{255,76,76}
\definecolor{green2.5}{RGB}{54, 247, 54}
\definecolor{blue2.5}{RGB}{51, 189, 255}
\definecolor{yellow2.5}{RGB}{255, 242, 52}


\definecolor{red3}{RGB}{255, 51, 51}
\definecolor{green3}{RGB}{0, 240, 0}
\definecolor{blue3}{RGB}{9, 175, 255}
\definecolor{yellow3}{RGB}{255, 234, 0}

\definecolor{red3.5}{RGB}{229, 25, 25}
\definecolor{green3.5}{RGB}{0, 194, 0}
\definecolor{blue3.5}{RGB}{4, 143, 209}
\definecolor{yellow3.5}{RGB}{255,220,0}

\definecolor{red4}{RGB}{204, 0, 0}
\definecolor{green4}{RGB}{0, 149, 0}
\definecolor{blue4}{RGB}{0, 111, 164}
\definecolor{yellow4}{RGB}{255, 206, 0}


\definecolor{noseve}{RGB}{242,242,242}

\newcommand{\camod}[1]{\frac{\ZZ}{#1 \ZZ}}
\newcommand{\modm}[1]{\text{ mod } #1}
\newcommand{\campm}[1]{\frac{\ZZ}{m\ZZ}}

\usepackage{epigraph}
\renewcommand{\epigraphsize}{\scriptsize}
\renewcommand{\epigraphwidth}{60ex}


\definecolor{dcol0}{HTML}{C8E6C9}
\definecolor{dcol1}{HTML}{D4E9B3}
\definecolor{dcol2}{HTML}{E5ED9A}
\definecolor{dcol3}{HTML}{FFF59D}
\definecolor{dcol4}{HTML}{FFE082}
\definecolor{dcol5}{HTML}{FFCC80}
\definecolor{dcol6}{HTML}{FFAB91}
\definecolor{dcol7}{HTML}{F49890}
\definecolor{dcol8}{HTML}{E57373}
\definecolor{dcol9}{HTML}{D32F2F}

\makeatletter
\newcommand{\getcolorname}[1]{dcol#1}
\makeatother

\newcommand{\dif}[1]{%
    \edef\colorindex{\number\fpeval{floor(#1)}}%
    \edef\fulltext{#1}%
    \colorbox{\getcolorname{\colorindex}}{%
        \ifnum\colorindex>8
            \textbf{\textcolor{white}{\,\fulltext\,}}%
        \else
            \textbf{\textcolor{black}{\,\fulltext\,}}%
        \fi
    }%
}
% Variable para dificultad (inicial 0)
\newcommand{\thmdifficulty}{0}

% Comando para asignar dificultad antes del problema
\newcommand{\problemdiff}[1]{\renewcommand{\thmdifficulty}{#1}}

% Estilo del problema que incluye dificultad antes del título
\declaretheoremstyle[
    headfont=\color{blue!40!black}\normalfont\bfseries,
    headformat={%
      \dif{\thmdifficulty}\quad \NAME~\NUMBER\ifx\relax\EMPTY\relax\else\ \NOTE\fi
    },
    postheadspace=1em,
    spaceabove=8pt,
    spacebelow=8pt,
    bodyfont=\normalfont
]{problemstyle}

    \declaretheorem[style=problemstyle,name=Problema,sibling=theorem]{problema}
    \declaretheorem[style=problemstyle,name=Problema,numbered=no]{problema*}

\title {8.- DOM y JS (Actividad 1)}
\subtitle{Programación WEB I \\ Centro de Enseñanza Tecnica Industrial}
\date{5 de Octubre de 2025}
\author{Emmanuel Buenrostro 22300891 7F1}


\begin{document}

\maketitle


\begin{center}
   \includegraphics[scale=0.15]{../cetilogo.jpg} 
\end{center}
\newpage
%-------------------- RESUMEN --------------------
\section{Resumen}
El lenguaje JavaScript es uno de los pilares fundamentales del desarrollo web moderno. Su principal uso es dotar de interactividad y dinamismo a las páginas web, permitiendo manipular elementos, validar formularios, y responder a eventos del usuario.  
En este trabajo se aborda su definición, sus principales características, y un ejemplo del uso del DOM (Document Object Model), el cual es una de las partes más importantes de este lenguaje.  

%-------------------- DESARROLLO --------------------
\section{Desarrollo}

\subsection*{Definición de JavaScript}
JavaScript es un lenguaje de programación interpretado, orientado a objetos y basado en prototipos, ampliamente utilizado para el desarrollo web. Fue creado por Brendan Eich en 1995 y desde entonces se ha convertido en un estándar gracias a su inclusión en todos los navegadores modernos.  
Este lenguaje permite que las páginas web sean dinámicas, interactivas y respondan a las acciones del usuario sin necesidad de recargar la página.

\subsection{Características principales}
\begin{itemize}
    \item Es un lenguaje interpretado, lo que significa que no necesita compilación.
    \item Se ejecuta del lado del cliente, aunque también puede ejecutarse del lado del servidor con Node.js.
    \item Tiene una sintaxis similar a C y Java, lo que facilita su aprendizaje.
    \item Permite manipular elementos HTML y CSS a través del DOM.
\end{itemize}

\subsection{Ejemplo del DOM}
El DOM (Document Object Model) es una representación estructurada del documento HTML. Permite que JavaScript acceda y modifique el contenido, estructura y estilo de una página web.

\noindent
Ejemplo básico:
\begin{verbatim}
<!DOCTYPE html>
<html>
<head>
    <title>Ejemplo DOM</title>
</head>
<body>
    <h1 id="titulo">Hola Mundo</h1>
    <button onclick="cambiarTexto()">Cambiar texto</button>

    <script>
        function cambiarTexto() {
            document.getElementById("titulo").innerHTML = "Texto modificado con JavaScript";
        }
    </script>
</body>
</html>
\end{verbatim}

En este ejemplo, el código JavaScript utiliza el método \texttt{getElementById} para acceder al elemento con el identificador \texttt{titulo} y cambiar su contenido al hacer clic en el botón.  

%-------------------- CONCLUSIÓN --------------------
\section{Conclusión}
JavaScript es un lenguaje esencial para el desarrollo web actual. Gracias al DOM, permite que los sitios sean dinámicos, interactivos y más atractivos para los usuarios. Además, su versatilidad lo ha llevado a ser utilizado no solo en navegadores, sino también en servidores y aplicaciones móviles.  

%-------------------- BIBLIOGRAFÍA --------------------
\section{Bibliografía}
\begin{itemize}
    \item Mozilla Developer Network (MDN). (s.f.). \textit{JavaScript}. Recuperado de: \url{https://developer.mozilla.org/es/docs/Web/JavaScript}
    \item W3Schools. (s.f.). \textit{JavaScript Tutorial}. Recuperado de: \url{https://www.w3schools.com/js/}
    \item GeeksforGeeks. (2023). \textit{What is DOM in JavaScript?}. Recuperado de: \url{https://www.geeksforgeeks.org/what-is-dom-in-javascript/}
\end{itemize}

\end{document}
