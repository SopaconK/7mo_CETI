\documentclass[11pt]{scrartcl}

\usepackage[sexy]{evan}
\usepackage{pgfplots}
\pgfplotsset{compat=1.15}
\usepackage{mathrsfs}
\usetikzlibrary{arrows}
\usepackage{graphics}
\usepackage{tikz}
\usepackage{ amssymb }
\usepackage[dvipsnames]{xcolor}
\usepackage[utf8]{inputenc}
\usepackage{longtable}
\usepackage{ragged2e}
\usepackage{listings}


\definecolor{noseve}{RGB}{242,242,242}

\newcommand{\camod}[1]{\frac{\ZZ}{#1 \ZZ}}
\newcommand{\modm}[1]{\text{ mod } #1}
\newcommand{\campm}[1]{\frac{\ZZ}{m\ZZ}}

\usepackage{epigraph}
\renewcommand{\epigraphsize}{\scriptsize}
\renewcommand{\epigraphwidth}{60ex}


\definecolor{dcol0}{HTML}{C8E6C9}
\definecolor{dcol1}{HTML}{D4E9B3}
\definecolor{dcol2}{HTML}{E5ED9A}
\definecolor{dcol3}{HTML}{FFF59D}
\definecolor{dcol4}{HTML}{FFE082}
\definecolor{dcol5}{HTML}{FFCC80}
\definecolor{dcol6}{HTML}{FFAB91}
\definecolor{dcol7}{HTML}{F49890}
\definecolor{dcol8}{HTML}{E57373}
\definecolor{dcol9}{HTML}{D32F2F}

\makeatletter
\newcommand{\getcolorname}[1]{dcol#1}
\makeatother

\newcommand{\dif}[1]{%
    \edef\colorindex{\number\fpeval{floor(#1)}}%
    \edef\fulltext{#1}%
    \colorbox{\getcolorname{\colorindex}}{%
        \ifnum\colorindex>8
            \textbf{\textcolor{white}{\,\fulltext\,}}%
        \else
            \textbf{\textcolor{black}{\,\fulltext\,}}%
        \fi
    }%
}
% Variable para dificultad (inicial 0)
\newcommand{\thmdifficulty}{0}

% Comando para asignar dificultad antes del problema
\newcommand{\problemdiff}[1]{\renewcommand{\thmdifficulty}{#1}}

% Estilo del problema que incluye dificultad antes del título
\declaretheoremstyle[
    headfont=\color{blue!40!black}\normalfont\bfseries,
    headformat={%
      \dif{\thmdifficulty}\quad \NAME~\NUMBER\ifx\relax\EMPTY\relax\else\ \NOTE\fi
    },
    postheadspace=1em,
    spaceabove=8pt,
    spacebelow=8pt,
    bodyfont=\normalfont
]{problemstyle}

    \declaretheorem[style=problemstyle,name=Problema,sibling=theorem]{problema}
    \declaretheorem[style=problemstyle,name=Problema,numbered=no]{problema*}

%\usepackage[
%backend=biber,
%style=alphabetic,
%sorting=ynt
%]{biblatex}
%\addbibresource{referencias.bib}

\newcommand{\indicacion}[1]{\noindent\textit{\small #1}}


\title {Practica 1: Funcionamiento del ADC}
\subtitle{Sistemas de Medicion y Control 18MPEDS0730 \\ Ago-Dic 2025 \\ Centro de Enseñanza Tecnica Industrial Plantel Colomos\\Tgo. en Desarrollo de Software \\ Academia: Sistemas Electrónicos\\Profesor: Diana Marisol Figueroa Flores }
\date{27 de Agosto de 2025}
\author{Emmanuel Buenrostro 22300891 7F1 \\ \and Emiliano Arzate 22300929 7F1 \\}


\begin{document}

\maketitle
\begin{center}
   \includegraphics[scale=0.15]{../../cetilogo.jpg} 
\end{center}
\newpage


\section{Objetivo}

\textbf{Objetivo General:} Verificar el funcionamiento básico de un ADC de 8 bits y 10 bits. \\


\textbf{Objetivos Específicos:} Identificar la resolución utilizando un sistema de adquisición de datos, considerando una palabra de 8 bits y 10 bits, anexando un acoplamiento digital.

\section{Desarrollo Teórico}

\subsection{Resumen }

\indicacion{
    Elaborar un resumen sobre los tipos de ADC´s. Anexar las referencias bibliográficas una referencia deberá ser virtual y la otra de un libro, considerando el formato APA correspondiente al tipo de referencia.
}


\subsection{Material}

\indicacion{
    Anotar el material y equipo para llevar a cabo la práctica agregando los valores teóricos.
}

\subsection{Caracteristicas Electricas de los Componentes}
\indicacion{
    Anexar características eléctricas de todos los componentes a utilizar, así como los voltajes y corrientes máximas de trabajo, distribución de terminales, etc.
}

\subsection{Diagrama a Bloques}

\subsection{Calculos}
\indicacion{
    Realizar los cálculos correspondientes para la resolución, considerando la parte alta y la parte baja de direcciones para la resolución correspondiente.
}

\subsection{Tabla de Mediciones Teoricas}
\indicacion{
    Agregar la tabla de mediciones Teóricas, con sus 10 voltajes a verificar.
}

\begin{center}
\begin{tabular}{|c|c|}
\hline
\textbf{V analógico}& \textbf{Equivalente en binario}\\
\hline
& \\
\hline
& \\
\hline
& \\
\hline
& \\
\hline
& \\
\hline
& \\
\hline
& \\
\hline
& \\
\hline
& \\
\hline
& \\
\hline
\end{tabular}
\end{center}


\section{Desarrollo Practico}
\subsection{Proceso}
\indicacion{
    Describe los pasos desde el inicio de la elaboración de la práctica hasta el término de la misma.
}

\subsection{Diagrama Eléctrico}
\indicacion{
    Dibujar el Diagrama eléctrico utilizando algún programa para elaborar circuitos electrónicos, sin olvidar el valor de los componentes reales.
}

\subsection{Tabla Comparativa}

\indicacion{
    Agregar la tabla comparativa de valores teóricos y prácticos, en caso de tener gráficas anexarlas.
}

\begin{center}
\begin{tabular}{|c|c|p{3cm}|c|}
\hline
\textbf{V analógico}&\textbf{Equivalente en Binario} &\textbf{Voltaje en el Sistema de adquisición de Datos} & \textbf{Equivalente en Decimal}\\
\hline
& & & \\
\hline
& & & \\
\hline
& & & \\
\hline
& & & \\
\hline
& & & \\
\hline
& & & \\
\hline
& & & \\
\hline
& & & \\
\hline
& & & \\
\hline
& & & \\
\hline
\end{tabular}
\end{center}

\section{Observaciones y Conclusiones}

\subsection{Observaciones}
\indicacion{
    Elaborar las observaciones correspondientes.
}

\subsection{Conclusiones Personales}
\indicacion{
    Realizar las conclusiones correspondientes de forma personal anexando usos y aplicaciones de lo aprendido.
}


  
%\nocite{*}

%\printbibliography[
%heading=bibintoc,
%title={ . }
%]
    \end{document}