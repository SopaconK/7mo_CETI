\documentclass[11pt]{scrartcl}

\usepackage[sexy]{evan}
\usepackage{pgfplots}
\pgfplotsset{compat=1.15}
\usepackage{mathrsfs}
\usetikzlibrary{arrows}
\usepackage{graphics}
\usepackage{tikz}
\usepackage{ amssymb }
\usepackage[dvipsnames]{xcolor}
\usepackage[utf8]{inputenc}
\usepackage{longtable}
\usepackage{ragged2e}
\usepackage{listings}


\definecolor{noseve}{RGB}{242,242,242}

\newcommand{\camod}[1]{\frac{\ZZ}{#1 \ZZ}}
\newcommand{\modm}[1]{\text{ mod } #1}
\newcommand{\campm}[1]{\frac{\ZZ}{m\ZZ}}

\usepackage{epigraph}
\renewcommand{\epigraphsize}{\scriptsize}
\renewcommand{\epigraphwidth}{60ex}


\definecolor{dcol0}{HTML}{C8E6C9}
\definecolor{dcol1}{HTML}{D4E9B3}
\definecolor{dcol2}{HTML}{E5ED9A}
\definecolor{dcol3}{HTML}{FFF59D}
\definecolor{dcol4}{HTML}{FFE082}
\definecolor{dcol5}{HTML}{FFCC80}
\definecolor{dcol6}{HTML}{FFAB91}
\definecolor{dcol7}{HTML}{F49890}
\definecolor{dcol8}{HTML}{E57373}
\definecolor{dcol9}{HTML}{D32F2F}

\makeatletter
\newcommand{\getcolorname}[1]{dcol#1}
\makeatother

\newcommand{\dif}[1]{%
    \edef\colorindex{\number\fpeval{floor(#1)}}%
    \edef\fulltext{#1}%
    \colorbox{\getcolorname{\colorindex}}{%
        \ifnum\colorindex>8
            \textbf{\textcolor{white}{\,\fulltext\,}}%
        \else
            \textbf{\textcolor{black}{\,\fulltext\,}}%
        \fi
    }%
}
% Variable para dificultad (inicial 0)
\newcommand{\thmdifficulty}{0}

% Comando para asignar dificultad antes del problema
\newcommand{\problemdiff}[1]{\renewcommand{\thmdifficulty}{#1}}

% Estilo del problema que incluye dificultad antes del título
\declaretheoremstyle[
    headfont=\color{blue!40!black}\normalfont\bfseries,
    headformat={%
      \dif{\thmdifficulty}\quad \NAME~\NUMBER\ifx\relax\EMPTY\relax\else\ \NOTE\fi
    },
    postheadspace=1em,
    spaceabove=8pt,
    spacebelow=8pt,
    bodyfont=\normalfont
]{problemstyle}

    \declaretheorem[style=problemstyle,name=Problema,sibling=theorem]{problema}
    \declaretheorem[style=problemstyle,name=Problema,numbered=no]{problema*}

%\usepackage[
%backend=biber,
%style=alphabetic,
%sorting=ynt
%]{biblatex}
%\addbibresource{referencias.bib}

\newcommand{\indicacion}[1]{\noindent\textit{\small #1}}


\title {Practica 3: Caracteristicas Dinámicas de los Instrumentos}
\subtitle{Sistemas de Medicion y Control 18MPEDS0730 \\ Ago-Dic 2025 \\ Centro de Enseñanza Tecnica Industrial Plantel Colomos\\Tgo. en Desarrollo de Software \\ Academia: Sistemas Electrónicos\\Profesor: Diana Marisol Figueroa Flores }
\date{27 de Agosto de 2025}
\author{Emmanuel Buenrostro 22300891 7F1 \\ \and Emiliano Arzate 22300929 7F1 \\}


\begin{document}

\maketitle
\begin{center}
   \includegraphics[scale=0.15]{../../cetilogo.jpg} 
\end{center}
\newpage


\section{Objetivo}

\textbf{Objetivo General:}
Reconocer las características dinámicas de algunos instrumentos de medición.
\\


\textbf{Objetivos Específicos:} 
Identificar las características dinámicas de un multímetro y un osciloscopio, mediante la utilización de manuales correspondiente a cada tipo de instrumento de medición.

\section{Desarrollo Teórico}

\subsection{Resumen }

\indicacion{
   Elaborar un resumen las características dinámicas de los instrumentos de medición. Anexar las referencias bibliográficas una referencia deberá ser virtual y la otra de un libro, considerando el formato APA correspondiente al tipo de referencia.
}

\subsection{Material}

\indicacion{
Anotar el material y equipo para llevar a cabo la práctica agregando los valores teóricos.}

\subsection{Caracteristicas Electricas de los Componentes}
\indicacion{
   Anexar características eléctricas de todos los componentes a utilizar, así como los voltajes y corrientes máximas de trabajo, distribución de terminales, etc.
}

\subsection{Caracteristicas Dinámicas de los Instrumentos}
\indicacion{
  Anexar características dinámicas de los diferentes tipos de instrumentos, así como las características eléctricas de todos los componentes a utilizar los voltajes y corrientes máximas de trabajo, distribución de terminales, etc.
}


\subsection{Diagrama a Bloques}

\subsection{Diagrama Electrico}

\indicacion{
    Dibujar el Diagrama eléctrico utilizando algún programa para elaborar circuitos electrónicos, sin olvidar el valor de los componentes reales.
}

\subsection{Calculos}
\indicacion{
Realizar los cálculos correspondientes para los circuitos: emisor común por divisor de voltaje, amplificador a pequeña señal con transistores tanto en corriente directa como en corriente alterna, circuito inversor o no inversor en corriente directa y corriente alterna sin olvidar anexar las señales teóricas con sus diagramas y parámetros correspondientes. Nota anexa el número de horas que consideres necesario para la realización de los cálculos teóricos.
}


% --------------------------------------------------------------------
% Circuito 1: Amplificador Inversor en CD
% --------------------------------------------------------------------
\subsubsection{Circuito 1: Amplificador Inversor con TL082 (Análisis en CD)}

Para un amplificador operacional ideal, la diferencia de potencial entre sus entradas es nula. Dado que la entrada no inversora ($V_+$) está conectada a tierra (0V), se crea una \textbf{tierra virtual} en la entrada inversora ($V_-$).
\begin{equation}
    V_+ = V_- = 0 \, V
\end{equation}
La corriente que fluye a través de $R_1$ es igual a la que fluye por $R_f$, ya que la impedancia de entrada del op-amp es infinita.
\begin{align*}
    I_1 &= I_f \\
    \frac{V_{in} - V_-}{R_1} &= \frac{V_- - V_{out}}{R_f}
\end{align*}
Sustituyendo $V_- = 0V$:
\begin{equation}
    \frac{V_{in}}{R_1} = \frac{-V_{out}}{R_f}
\end{equation}
La ganancia en lazo cerrado para CD ($A_{v,dc}$) es:
\begin{equation}
    A_{v,dc} = \frac{V_{out}}{V_{in}} = -\frac{R_f}{R_1}
\end{equation}


\begin{align*}
    A_{v,dc} &= -\frac{10 \, k\Omega}{1 \, k\Omega} = -10 \\
    V_{out} &= A_{v,dc} \times V_{in} = -10 \times 0.5 \, V = \mathbf{-5 \, V}
\end{align*}


% --------------------------------------------------------------------
% Circuito 2: Amplificador Inversor en CA
% --------------------------------------------------------------------
\subsubsection{Circuito 2: Amplificador Inversor con TL082 (Análisis en CA)}

Dado que el circuito solo contiene resistencias, su comportamiento es independiente de la frecuencia (en el rango de operación del op-amp). La ganancia de voltaje en CA ($A_{v,ac}$) es idéntica a la de CD.
\begin{equation}
    A_{v,ac} = -\frac{R_f}{R_1} = -10
\end{equation}

%\subsection{Cálculo Específico}
La señal de salida $v_{out}(t)$ será la señal de entrada multiplicada por la ganancia.
\begin{align*}
    v_{out}(t) &= A_{v,ac} \times v_{in}(t) \\
    v_{out}(t) &= -10 \times (20 \, mV \sin(2\pi \cdot 1000t)) \\
    v_{out}(t) &= -200 \, mV \sin(2\pi \cdot 1000t)
\end{align*}




% --------------------------------------------------------------------
% Circuito 3: Amplificador BJT en CD (Polarización)
% --------------------------------------------------------------------
\subsubsection{Circuito 3: Amplificador BJT 2N2222 (Análisis en CD)}

Se formalizan los cálculos de la imagen de referencia.
\begin{enumerate}
    \item \textbf{Resistencia de Emisor ($R_E$):} Se establece $V_E \approx 0.1 V_{CC} = 1.2 \, V$.
    \begin{equation*}
        R_E = \frac{V_E}{I_E} \approx \frac{V_E}{I_{C_Q}} = \frac{1.2 \, V}{0.002 \, A} = 600 \, \Omega \rightarrow \text{Valor comercial: } \mathbf{680 \, \Omega}
    \end{equation*}

    \item \textbf{Resistencia de Colector ($R_C$):}
    \begin{equation*}
        R_C = \frac{V_{CC} - V_{CE_Q} - V_E}{I_{C_Q}} = \frac{12 \, V - 6 \, V - 1.2 \, V}{0.002 \, A} = \frac{4.8 \, V}{0.002 \, A} = 2.4 \, k\Omega \rightarrow \text{Valor comercial: } \mathbf{2.2 \, k\Omega}
    \end{equation*}
    
    \item \textbf{Voltaje de Base ($V_B$):} Se asume $V_{BE} = 0.65 \, V$ (según la imagen).
    \begin{equation*}
        V_B = V_E + V_{BE} = 1.2 \, V + 0.65 \, V = 1.85 \, V
    \end{equation*}
    
    \item \textbf{Corrientes ($I_B, I_{div}$):}
    \begin{align*}
        I_B &= \frac{I_{C_Q}}{\beta} = \frac{2 \, mA}{50} = 40 \, \mu A \\
        I_{div} &= 10 \times I_B = 10 \times 40 \, \mu A = 400 \, \mu A
    \end{align*}
    
    \item \textbf{Resistencias del Divisor ($R_1, R_2$):}
    \begin{align*}
        R_2 &= \frac{V_B}{I_{div}} = \frac{1.85 \, V}{0.0004 \, A} = 4.625 \, k\Omega \rightarrow \text{Valor comercial: } \mathbf{4.7 \, k\Omega} \\
        R_1 &= \frac{V_{CC} - V_B}{I_{div} + I_B} = \frac{12 \, V - 1.85 \, V}{400 \, \mu A + 40 \, \mu A} = \frac{10.15 \, V}{0.00044 \, A} \approx 23 \, k\Omega \rightarrow \text{Valor comercial: } \mathbf{24 \, k\Omega} \text{ o } \mathbf{27 \, k\Omega}
    \end{align*}
\end{enumerate}

%\subsection{Punto de Operación Teórico (Q-point)}
Los valores de CD esperados en el circuito son $I_{C_Q} \approx 2 \, mA$ y $V_{CE_Q} \approx 6 \, V$.



% --------------------------------------------------------------------
% Circuito 4: Amplificador BJT en CA
% --------------------------------------------------------------------
\subsubsection{Circuito 4: Amplificador BJT 2N2222 (Análisis en CA)}

\begin{enumerate}
    \item \textbf{Resistencia interna de emisor ($r_e$):}
    \begin{equation*}
        r_e = \frac{V_T}{I_E} \approx \frac{26 \, mV}{I_{C_Q}} = \frac{26 \, mV}{2 \, mA} = 13 \, \Omega
    \end{equation*}
    
    \item \textbf{Impedancia de Entrada del Amplificador ($Z_{in}$):}
    El capacitor de bypass $C_E$ cortocircuita $R_E$ para la señal de CA.
    \begin{align*}
        Z_{base} &= \beta \times r_e = 50 \times 13 \, \Omega = 650 \, \Omega \\
        Z_{in} &= R_1 || R_2 || Z_{base} = 27 \, k\Omega || 4.7 \, k\Omega || 650 \, \Omega \approx \mathbf{550 \, \Omega}
    \end{align*}
    
    \item \textbf{Impedancia de Salida del Amplificador ($Z_{out}$):}
    \begin{equation*}
        Z_{out} \approx R_C = \mathbf{2.2 \, k\Omega}
    \end{equation*}

    \item \textbf{Ganancia de Voltaje ($A_v$):} La ganancia se ve afectada por la resistencia de carga $R_L$.
    \begin{align*}
        R'_{C} &= R_C || R_L = 2.2 \, k\Omega || 10 \, k\Omega = \frac{2.2 \times 10}{2.2 + 10} \, k\Omega \approx 1.8 \, k\Omega \\
        A_v &= -\frac{R'_{C}}{r_e} = -\frac{1800 \, \Omega}{13 \, \Omega} \approx \mathbf{-138.5}
    \end{align*}
\end{enumerate}

\textbf{Resultado Esperado para una Entrada de 10 mV}
Se aplica una señal $v_{in}(t) = 10 \, mV \sin(\omega t)$.
\begin{align*}
    v_{out}(t) &= A_v \times v_{in}(t) \\
    v_{out}(t) &= -138.5 \times (10 \, mV \sin(\omega t)) \\
    v_{out}(t) &= -1385 \, mV \sin(\omega t) = \mathbf{-1.385 \, V \sin(\omega t)}
\end{align*}
Se espera una señal de salida senoidal, sin distorsión, con una amplitud de \textbf{1.385 V} e invertida en fase 180° con respecto a la entrada.


\subsection{Tabla de Valores Teoricos de circuito emisor común con CD}
\indicacion{
Agregar la tabla de valores teóricos de circuito emisor común con divisor de voltaje del amplificador por divisor de voltaje en corriente directa
}

\begin{center}
\begin{tabular}{|p{3cm}|c|c|c|}
\hline
\textbf{Ganancia del transistor hfe}& \textbf{Corriente de base ($I_b$)} & \textbf{Corriente de Colector ($I_C$)} & \textbf{Voltaje de CE}\\
\hline
& & & \\[4px]
\hline
& & & \\[4px]
\hline
& & & \\[4px]
\hline
& & & \\[4px]
\hline
& & & \\[4px]
\hline
\end{tabular}
\end{center}


\section{Señales de entrada y salida a CA}
\indicacion{
    Anexar las señales de entrada y salida a corriente alterna con sus parámetros de corriente, voltaje, periodo y frecuencia.
}

\begin{center}
\begin{tabular}{|p{6cm}|p{6cm}|}
\hline
&  \\[100px]
\hline
\end{tabular}
\end{center}



\subsection{Tabla de Valores Teoricos del Amplificador en CD}
\indicacion{
Agregar la tabla de valores teóricos de circuito emisor común con divisor de voltaje del amplificador por divisor de voltaje en corriente directa
}

\begin{center}
\begin{tabular}{|p{3cm}|c|c|c|}
\hline
\textbf{Voltaje de Entrada}& \textbf{Voltaje de Salida} & \textbf{Voltaje Diferencial} & \textbf{$+V_{\text{sat}}, -V_{\text{sat}}$}\\
\hline
& & & \\[4px]
\hline
& & & \\[4px]
\hline
& & & \\[4px]
\hline
& & & \\[4px]
\hline
& & & \\[4px]
\hline
\end{tabular}
\end{center}


\section{Señales de entrada y salida a CA}
\indicacion{
    Anexar las señales de entrada y salida a corriente alterna con sus parámetros de corriente, voltaje, periodo y frecuencia.
}

\begin{center}
\begin{tabular}{|p{6cm}|p{6cm}|}
\hline
&  \\[100px]
\hline
\end{tabular}
\end{center}

\section{Desarrollo Practico}


\subsection{Tabla de Valores Practicos de circuito emisor común con CD}
\indicacion{
Agregar la tabla de valores teóricos de circuito emisor común con divisor de voltaje del amplificador por divisor de voltaje en corriente directa
}

\begin{center}
\begin{tabular}{|p{3cm}|c|c|c|}
\hline
\textbf{Ganancia del transistor hfe}& \textbf{Corriente de base ($I_b$)} & \textbf{Corriente de Colector ($I_C$)} & \textbf{Voltaje de CE}\\
\hline
& & & \\[4px]
\hline
& & & \\[4px]
\hline
& & & \\[4px]
\hline
& & & \\[4px]
\hline
& & & \\[4px]
\hline
\end{tabular}
\end{center}


\section{Señales de entrada y salida a CA}
\indicacion{
    Anexar las señales de entrada y salida a corriente alterna con sus parámetros de corriente, voltaje, periodo y frecuencia.
}

\begin{center}
\begin{tabular}{|p{6cm}|p{6cm}|}
\hline
&  \\[100px]
\hline
\end{tabular}
\end{center}



\subsection{Tabla de Valores Practicos del Amplificador en CD}
\indicacion{
Agregar la tabla de valores practicos de circuito emisor común con divisor de voltaje del amplificador por divisor de voltaje en corriente directa
}

\begin{center}
\begin{tabular}{|p{3cm}|c|c|c|}
\hline
\textbf{Voltaje de Entrada}& \textbf{Voltaje de Salida} & \textbf{Voltaje Diferencial} & \textbf{$+V_{\text{sat}}, -V_{\text{sat}}$}\\
\hline
5V & -11.3V & 24V&  10.54V, -11.3V\\[4px]
\hline
5V & -10.48V & 24V& 11V, -10.48V\\[4px]
\hline
5V & -11.29V & 24V& +10.53V, -11.29V\\[4px]
\hline
5V & -10.48V& 24V& +10.99V, -10.48V\\[4px]
\hline
5V & -11.30V& 24V& +10.54V , -11.30V\\[4px]
\hline
\end{tabular}
\end{center}


\section{Señales de entrada y salida a CA}
\indicacion{
    Anexar las señales de entrada y salida a corriente alterna con sus parámetros de corriente, voltaje, periodo y frecuencia.
}

\begin{center}
\begin{tabular}{|p{6cm}|p{6cm}|}
\hline
&  \\[100px]
\hline
\end{tabular}
\end{center}




\section{Observaciones y Conclusiones}

\subsection{Observaciones}
\indicacion{
    Elaborar las observaciones correspondientes.
}

\subsection{Conclusiones Personales}
\indicacion{
    Realizar las conclusiones correspondientes de forma personal anexando usos y aplicaciones de lo aprendido.
}


  
%\nocite{*}

%\printbibliography[
%heading=bibintoc,
%title={ . }
%]
    \end{document}