\documentclass[11pt]{scrartcl}

\usepackage[sexy]{evan}
\usepackage{pgfplots}
\pgfplotsset{compat=1.15}
\usepackage{mathrsfs}
\usetikzlibrary{arrows}
\usepackage{graphics}
\usepackage{tikz}
\usepackage{ amssymb }
\usepackage[dvipsnames]{xcolor}
\usepackage[utf8]{inputenc}
\usepackage{longtable}
\usepackage{ragged2e}
\usepackage{listings}


\definecolor{noseve}{RGB}{242,242,242}

\newcommand{\camod}[1]{\frac{\ZZ}{#1 \ZZ}}
\newcommand{\modm}[1]{\text{ mod } #1}
\newcommand{\campm}[1]{\frac{\ZZ}{m\ZZ}}

\usepackage{epigraph}
\renewcommand{\epigraphsize}{\scriptsize}
\renewcommand{\epigraphwidth}{60ex}


\definecolor{dcol0}{HTML}{C8E6C9}
\definecolor{dcol1}{HTML}{D4E9B3}
\definecolor{dcol2}{HTML}{E5ED9A}
\definecolor{dcol3}{HTML}{FFF59D}
\definecolor{dcol4}{HTML}{FFE082}
\definecolor{dcol5}{HTML}{FFCC80}
\definecolor{dcol6}{HTML}{FFAB91}
\definecolor{dcol7}{HTML}{F49890}
\definecolor{dcol8}{HTML}{E57373}
\definecolor{dcol9}{HTML}{D32F2F}

\makeatletter
\newcommand{\getcolorname}[1]{dcol#1}
\makeatother

\newcommand{\dif}[1]{%
    \edef\colorindex{\number\fpeval{floor(#1)}}%
    \edef\fulltext{#1}%
    \colorbox{\getcolorname{\colorindex}}{%
        \ifnum\colorindex>8
            \textbf{\textcolor{white}{\,\fulltext\,}}%
        \else
            \textbf{\textcolor{black}{\,\fulltext\,}}%
        \fi
    }%
}
% Variable para dificultad (inicial 0)
\newcommand{\thmdifficulty}{0}

% Comando para asignar dificultad antes del problema
\newcommand{\problemdiff}[1]{\renewcommand{\thmdifficulty}{#1}}

% Estilo del problema que incluye dificultad antes del título
\declaretheoremstyle[
    headfont=\color{blue!40!black}\normalfont\bfseries,
    headformat={%
      \dif{\thmdifficulty}\quad \NAME~\NUMBER\ifx\relax\EMPTY\relax\else\ \NOTE\fi
    },
    postheadspace=1em,
    spaceabove=8pt,
    spacebelow=8pt,
    bodyfont=\normalfont
]{problemstyle}

    \declaretheorem[style=problemstyle,name=Problema,sibling=theorem]{problema}
    \declaretheorem[style=problemstyle,name=Problema,numbered=no]{problema*}

%\usepackage[
%backend=biber,
%style=alphabetic,
%sorting=ynt
%]{biblatex}
%\addbibresource{referencias.bib}

\newcommand{\indicacion}[1]{\noindent\textit{\small #1}}


\title {Practica 2: Practica del motor a pasos}
\subtitle{Unidad I: Motores a CD Tema 1.1 \\ Sistemas Embebidos II 18MPEDS0729 \\ Ago-Dic 2025 \\ Centro de Enseñanza Tecnica Industrial Plantel Colomos\\Tgo. en Desarrollo de Software \\ Academia: Sistemas Digitales \\Profesor: Antonio Lozano Gonzáles }
\date{3 de Septiembre de 2025}
\author{Emmanuel Buenrostro 22300891 7F1 \\ \and Emiliano Arzate 22300929 7F1 \\}


\begin{document}

\maketitle
\begin{center}
   \includegraphics[scale=0.15]{../../cetilogo.jpg} 
\end{center}
\newpage

\section{Objetivo}

Mandar la secuencia de movimiento de un motor a pasos, por los pines
del Arduino; para el control de un motor a pasos en medios pasos y pasos
normal.

\section{Desarrollo de la Práctica}

\subsection{Condiciones de la Práctica}

Realizar la secuencia de un motor a pasos. Deberá el usuario del
programa, poder elegir entre medios pasos y pasos completos o normales.
Una vez hecha la selección, la secuencia sera visible en el motor a
pasos. También podrá elegir el usuario, si la secuencia sera en el
sentido de las manecillas del reloj o al contrario y adjuntar el reporte
respectivo.


\subsection{Algoritmo o Diagrama de Flujo}

\begin{enumerate}
    \item Iniciar el programa y mostrar el menú de opciones al usuario.
    \item Solicitar al usuario seleccionar el tipo de secuencia:
    \begin{itemize}
        \item Medios pasos
        \item Pasos completos (normales)
    \end{itemize}
    \item Solicitar al usuario seleccionar el sentido de giro:
    \begin{itemize}
        \item Sentido de las manecillas del reloj
        \item Sentido contrario a las manecillas del reloj
    \end{itemize}
    \item Configurar la secuencia de activación de los bobinados del motor según las opciones elegidas.
    \item Ejecutar la secuencia en el motor a pasos, mostrando el movimiento correspondiente.
  %  \item Permitir al usuario adjuntar el reporte respectivo.
    \item Finalizar el programa.
\end{enumerate}

\subsection{ Código C}

\lstinputlisting[language=C]{Pr2/Pr2.ino}


\section{Observaciones y Conclusiones}

\begin{itemize}
    \item El parcial pasado ya habiamos usado un motor a pasos. 
    \item Se realizaron funciones separadas, ya que tener cada rutina de que bobinas prender era muy poco optimo.
    \item Estos motores sirven para ir avanzando poco a poco de "una", mas que todo seguido. 
\end{itemize}
  
%\nocite{*}

%\printbibliography[
%heading=bibintoc,
%title={ . }
%]
    \end{document}