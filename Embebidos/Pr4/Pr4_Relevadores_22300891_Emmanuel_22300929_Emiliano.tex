\documentclass[11pt]{scrartcl}

\usepackage[sexy]{evan}
\usepackage{pgfplots}
\pgfplotsset{compat=1.15}
\usepackage{mathrsfs}
\usetikzlibrary{arrows}
\usepackage{graphics}
\usepackage{tikz}
\usepackage{ amssymb }
\usepackage[dvipsnames]{xcolor}
\usepackage[utf8]{inputenc}
\usepackage{longtable}
\usepackage{ragged2e}
\usepackage{listings}


\definecolor{noseve}{RGB}{242,242,242}

\newcommand{\camod}[1]{\frac{\ZZ}{#1 \ZZ}}
\newcommand{\modm}[1]{\text{ mod } #1}
\newcommand{\campm}[1]{\frac{\ZZ}{m\ZZ}}

\usepackage{epigraph}
\renewcommand{\epigraphsize}{\scriptsize}
\renewcommand{\epigraphwidth}{60ex}


\definecolor{dcol0}{HTML}{C8E6C9}
\definecolor{dcol1}{HTML}{D4E9B3}
\definecolor{dcol2}{HTML}{E5ED9A}
\definecolor{dcol3}{HTML}{FFF59D}
\definecolor{dcol4}{HTML}{FFE082}
\definecolor{dcol5}{HTML}{FFCC80}
\definecolor{dcol6}{HTML}{FFAB91}
\definecolor{dcol7}{HTML}{F49890}
\definecolor{dcol8}{HTML}{E57373}
\definecolor{dcol9}{HTML}{D32F2F}

\makeatletter
\newcommand{\getcolorname}[1]{dcol#1}
\makeatother

\newcommand{\dif}[1]{%
    \edef\colorindex{\number\fpeval{floor(#1)}}%
    \edef\fulltext{#1}%
    \colorbox{\getcolorname{\colorindex}}{%
        \ifnum\colorindex>8
            \textbf{\textcolor{white}{\,\fulltext\,}}%
        \else
            \textbf{\textcolor{black}{\,\fulltext\,}}%
        \fi
    }%
}
% Variable para dificultad (inicial 0)
\newcommand{\thmdifficulty}{0}

% Comando para asignar dificultad antes del problema
\newcommand{\problemdiff}[1]{\renewcommand{\thmdifficulty}{#1}}

% Estilo del problema que incluye dificultad antes del título
\declaretheoremstyle[
    headfont=\color{blue!40!black}\normalfont\bfseries,
    headformat={%
      \dif{\thmdifficulty}\quad \NAME~\NUMBER\ifx\relax\EMPTY\relax\else\ \NOTE\fi
    },
    postheadspace=1em,
    spaceabove=8pt,
    spacebelow=8pt,
    bodyfont=\normalfont
]{problemstyle}

    \declaretheorem[style=problemstyle,name=Problema,sibling=theorem]{problema}
    \declaretheorem[style=problemstyle,name=Problema,numbered=no]{problema*}

%\usepackage[
%backend=biber,
%style=alphabetic,
%sorting=ynt
%]{biblatex}
%\addbibresource{referencias.bib}

\newcommand{\indicacion}[1]{\noindent\textit{\small #1}}


\title {Practica 4: Practica de relevadores}
\subtitle{Unidad I: Motores a CD Tema 1.1 \\ Sistemas Embebidos II 18MPEDS0729 \\ Ago-Dic 2025 \\ Centro de Enseñanza Tecnica Industrial Plantel Colomos\\Tgo. en Desarrollo de Software \\ Academia: Sistemas Digitales \\Profesor: Antonio Lozano Gonzáles }
\date{17 de Septiembre de 2025}
\author{Emmanuel Buenrostro 22300891 7F1 \\ \and Emiliano Arzate 22300929 7F1 \\}


\begin{document}

\maketitle
\begin{center}
   \includegraphics[scale=0.15]{../../cetilogo.jpg} 
\end{center}
\newpage

\section{Objetivo}

 Usar relevadores para el cambio de giro de un motor a CD y ver alguna aplicación diferente.
\section{Desarrollo de la Práctica}

\subsection{Condiciones de la Práctica}
la cual consiste en usar dos reveladores, para
poder hacer el cambio de giro de un motor a CD; deberán preguntar el
sentido de giro del motor y programar los cuatro porcentajes de energía
que se le entregara a dicho motor, dichas velocidades se entregaran
también en un tiempo determinado por el usuario, que es de máximo dos
minutos, 


\subsection{Algoritmo o Diagrama de Flujo}

\begin{enumerate}
    \item Al iniciar el sistema, se realiza la configuración de la pantalla LCD, el teclado matricial y los pines necesarios para controlar los relevadores y el motor de corriente directa. Se muestra un mensaje de bienvenida en la pantalla para indicar el inicio de la práctica.
    \item Posteriormente, el sistema solicita al usuario que seleccione el sentido de giro del motor, mostrando las opciones disponibles en la pantalla LCD. El usuario debe elegir entre giro a la izquierda o a la derecha utilizando el teclado.
    \item Una vez seleccionado el sentido de giro, el sistema pide al usuario que ingrese el tiempo de operación del motor, el cual debe estar en el rango de uno a ciento veinte segundos. El usuario introduce el valor deseado mediante el teclado y lo confirma.
    \item Con los datos ingresados, el sistema activa los relevadores para establecer el sentido de giro seleccionado y comienza a controlar el motor. La velocidad del motor se incrementa en cuatro etapas: primero al 25\%, luego al 50\%, después al 75\% y finalmente al 100\%, mostrando cada porcentaje en la pantalla LCD y manteniendo cada velocidad durante el tiempo especificado.
    \item Al finalizar el ciclo de velocidades, el sistema apaga el motor y los relevadores, mostrando un mensaje de finalización en la pantalla LCD. Finalmente, espera unos segundos antes de reiniciar el proceso para permitir una nueva operación.
\end{enumerate}

\subsection{ Código C}

\lstinputlisting[language=C]{p4_embebidos.ino}


\section{Observaciones y Conclusiones}

\begin{itemize}
    \item Estamos realizando el puente H pero por nuestra cuenta.
    \item Esta practica estuvo bastante compleja porque teniamos que ver donde/como poner los transisotres 
    \item Además usamos un transistor que aguanta mas voltaje de los que solemos usualmente usar. 
\end{itemize}
  
%\nocite{*}

%\printbibliography[
%heading=bibintoc,
%title={ . }
%]
    \end{document}