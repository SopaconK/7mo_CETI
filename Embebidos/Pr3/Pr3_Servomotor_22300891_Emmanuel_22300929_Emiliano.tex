\documentclass[11pt]{scrartcl}

\usepackage[sexy]{evan}
\usepackage{pgfplots}
\pgfplotsset{compat=1.15}
\usepackage{mathrsfs}
\usetikzlibrary{arrows}
\usepackage{graphics}
\usepackage{tikz}
\usepackage{ amssymb }
\usepackage[dvipsnames]{xcolor}
\usepackage[utf8]{inputenc}
\usepackage{longtable}
\usepackage{ragged2e}
\usepackage{listings}


\definecolor{noseve}{RGB}{242,242,242}

\newcommand{\camod}[1]{\frac{\ZZ}{#1 \ZZ}}
\newcommand{\modm}[1]{\text{ mod } #1}
\newcommand{\campm}[1]{\frac{\ZZ}{m\ZZ}}

\usepackage{epigraph}
\renewcommand{\epigraphsize}{\scriptsize}
\renewcommand{\epigraphwidth}{60ex}


\definecolor{dcol0}{HTML}{C8E6C9}
\definecolor{dcol1}{HTML}{D4E9B3}
\definecolor{dcol2}{HTML}{E5ED9A}
\definecolor{dcol3}{HTML}{FFF59D}
\definecolor{dcol4}{HTML}{FFE082}
\definecolor{dcol5}{HTML}{FFCC80}
\definecolor{dcol6}{HTML}{FFAB91}
\definecolor{dcol7}{HTML}{F49890}
\definecolor{dcol8}{HTML}{E57373}
\definecolor{dcol9}{HTML}{D32F2F}

\makeatletter
\newcommand{\getcolorname}[1]{dcol#1}
\makeatother

\newcommand{\dif}[1]{%
    \edef\colorindex{\number\fpeval{floor(#1)}}%
    \edef\fulltext{#1}%
    \colorbox{\getcolorname{\colorindex}}{%
        \ifnum\colorindex>8
            \textbf{\textcolor{white}{\,\fulltext\,}}%
        \else
            \textbf{\textcolor{black}{\,\fulltext\,}}%
        \fi
    }%
}
% Variable para dificultad (inicial 0)
\newcommand{\thmdifficulty}{0}

% Comando para asignar dificultad antes del problema
\newcommand{\problemdiff}[1]{\renewcommand{\thmdifficulty}{#1}}

% Estilo del problema que incluye dificultad antes del título
\declaretheoremstyle[
    headfont=\color{blue!40!black}\normalfont\bfseries,
    headformat={%
      \dif{\thmdifficulty}\quad \NAME~\NUMBER\ifx\relax\EMPTY\relax\else\ \NOTE\fi
    },
    postheadspace=1em,
    spaceabove=8pt,
    spacebelow=8pt,
    bodyfont=\normalfont
]{problemstyle}

    \declaretheorem[style=problemstyle,name=Problema,sibling=theorem]{problema}
    \declaretheorem[style=problemstyle,name=Problema,numbered=no]{problema*}

%\usepackage[
%backend=biber,
%style=alphabetic,
%sorting=ynt
%]{biblatex}
%\addbibresource{referencias.bib}

\newcommand{\indicacion}[1]{\noindent\textit{\small #1}}


\title {Practica 3: Practica de servomotor}
\subtitle{Unidad I: Motores a CD Tema 1.1 \\ Sistemas Embebidos II 18MPEDS0729 \\ Ago-Dic 2025 \\ Centro de Enseñanza Tecnica Industrial Plantel Colomos\\Tgo. en Desarrollo de Software \\ Academia: Sistemas Digitales \\Profesor: Antonio Lozano Gonzáles }
\date{3 de Septiembre de 2025}
\author{Emmanuel Buenrostro 22300891 7F1 \\ \and Emiliano Arzate 22300929 7F1 \\}


\begin{document}

\maketitle
\begin{center}
   \includegraphics[scale=0.15]{../../cetilogo.jpg} 
\end{center}
\newpage

\section{Objetivo}

Mandar la secuencia de movimiento a un servomotor, para poder posicionar en cualquier angulo, de entre 0 a 180 grados

\section{Desarrollo de la Práctica}

\subsection{Condiciones de la Práctica}

Realizar unas secuencias en un servomotor. Deberá el usuario del
programa, poder elegir desde uno hasta 24 movimientos del servomotor.
Una vez hecha la selección, la secuencia sera visible en el servomotor.
Recordar que cada secuencia es un angulo al que se desplazara el
servomotor, y dicho desplazamiento sera entre 0 y 180 grados, y que
deberá quedarse entre tres y cinco segundos en dicha secuencia, para
poder verlo y adjuntar el reporte respectivo. 
Recordar el uso de la
pantalla y el teclado para facilitar el acceso al control del servo y
para visualizar el angulo en que esta.


\subsection{Algoritmo o Diagrama de Flujo}

\begin{enumerate}
  \item Inicializa la pantalla LCD, el teclado matricial y el servo motor.
\item Solicita al usuario ingresar un número entre 1 y 24 usando el teclado; el número se confirma presionando '*'.
\item Muestra el número ingresado en la pantalla LCD.
\item Solicita al usuario ingresar una cantidad igual de ángulos, uno por uno, usando el teclado.
\item Almacena cada ángulo en un arreglo.
\item Mueve el servo motor a cada uno de los ángulos ingresados, mostrando el valor en la pantalla LCD y esperando 3 segundos entre cada movimiento.
\item Al finalizar, regresa el servo a la posición 0°.
\end{enumerate}

\subsection{ Código C}

\lstinputlisting[language=C]{Pr3_embebidos2.ino}


\section{Observaciones y Conclusiones}

\begin{itemize}
    \item El semestre pasado ya habiamos usado un servomotor 
    \item Nuestro servomotor es uno con mayor promedia que el prmoedio, entonces ocupaba mas voltaje 
    \item El cambio de 90° es el que mas potencia le costaba.
    \item El servomotor sirve para tener un mayor control sobre que tanto esta girando algo. 
\end{itemize}
  
%\nocite{*}

%\printbibliography[
%heading=bibintoc,
%title={ . }
%]
    \end{document}