\documentclass[11pt]{scrartcl}

\usepackage[sexy]{evan}
\usepackage{pgfplots}
\pgfplotsset{compat=1.15}
\usepackage{mathrsfs}
\usetikzlibrary{arrows}
\usepackage{graphics}
\usepackage{tikz}
\usepackage{ amssymb }
\usepackage[dvipsnames]{xcolor}
\usepackage[utf8]{inputenc}
\usepackage{longtable}
\usepackage{ragged2e}
\usepackage{listings}


\definecolor{noseve}{RGB}{242,242,242}

\newcommand{\camod}[1]{\frac{\ZZ}{#1 \ZZ}}
\newcommand{\modm}[1]{\text{ mod } #1}
\newcommand{\campm}[1]{\frac{\ZZ}{m\ZZ}}

\usepackage{epigraph}
\renewcommand{\epigraphsize}{\scriptsize}
\renewcommand{\epigraphwidth}{60ex}


\definecolor{dcol0}{HTML}{C8E6C9}
\definecolor{dcol1}{HTML}{D4E9B3}
\definecolor{dcol2}{HTML}{E5ED9A}
\definecolor{dcol3}{HTML}{FFF59D}
\definecolor{dcol4}{HTML}{FFE082}
\definecolor{dcol5}{HTML}{FFCC80}
\definecolor{dcol6}{HTML}{FFAB91}
\definecolor{dcol7}{HTML}{F49890}
\definecolor{dcol8}{HTML}{E57373}
\definecolor{dcol9}{HTML}{D32F2F}

\makeatletter
\newcommand{\getcolorname}[1]{dcol#1}
\makeatother

\newcommand{\dif}[1]{%
    \edef\colorindex{\number\fpeval{floor(#1)}}%
    \edef\fulltext{#1}%
    \colorbox{\getcolorname{\colorindex}}{%
        \ifnum\colorindex>8
            \textbf{\textcolor{white}{\,\fulltext\,}}%
        \else
            \textbf{\textcolor{black}{\,\fulltext\,}}%
        \fi
    }%
}
% Variable para dificultad (inicial 0)
\newcommand{\thmdifficulty}{0}

% Comando para asignar dificultad antes del problema
\newcommand{\problemdiff}[1]{\renewcommand{\thmdifficulty}{#1}}

% Estilo del problema que incluye dificultad antes del título
\declaretheoremstyle[
    headfont=\color{blue!40!black}\normalfont\bfseries,
    headformat={%
      \dif{\thmdifficulty}\quad \NAME~\NUMBER\ifx\relax\EMPTY\relax\else\ \NOTE\fi
    },
    postheadspace=1em,
    spaceabove=8pt,
    spacebelow=8pt,
    bodyfont=\normalfont
]{problemstyle}

    \declaretheorem[style=problemstyle,name=Problema,sibling=theorem]{problema}
    \declaretheorem[style=problemstyle,name=Problema,numbered=no]{problema*}

%\usepackage[
%backend=biber,
%style=alphabetic,
%sorting=ynt
%]{biblatex}
%\addbibresource{referencias.bib}

\newcommand{\indicacion}[1]{\noindent\textit{\small #1}}


\title {Practica 1: Práctica de control del motor a CD.}
\subtitle{Unidad I: Motores a CD Tema 1.1 \\ Sistemas Embebidos II 18MPEDS0729 \\ Ago-Dic 2025 \\ Centro de Enseñanza Tecnica Industrial Plantel Colomos\\Tgo. en Desarrollo de Software \\ Academia: Sistemas Digitales \\Profesor: Antonio Lozano Gonzáles }
\date{27 de Agosto de 2025}
\author{Emmanuel Buenrostro 22300891 7F1 \\ \and Emiliano Arzate 22300929 7F1 \\}


\begin{document}

\maketitle
\begin{center}
   \includegraphics[scale=0.15]{../../cetilogo.jpg} 
\end{center}
\newpage

\section{Objetivo}

Variar la velocidad de un motor a CD,  utilizando PWM y en cualquier tiempo programado.


\section{Desarrollo de la Práctica}

\subsection{Condiciones de la Práctica}

Realizar el control de un motor a CD. Deberá acelerar desde cero hasta
la velocidad máxima en un tiempo estipulado.  acompañado del reporte
respectivo. Utilizar el teclado para pedir el tiempo y, la pantalla para
mostrar las preguntas necesarias y, el porcentaje de aceleración del
motor.  \\
También deberá tener un instrumento o sensor que muestre las revoluciones por minuto del motor. Que se verán en la pantalla.
\subsection{Algoritmo o Diagrama de Flujo}

\begin{itemize}
\item En la inicialización se configura la pantalla LCD, el teclado matricial y los pines del motor y sensor.
Muestra una pantalla inicial pidiendo el tiempo de aceleración.


\item Ingreso de tiempo:

El usuario ingresa el tiempo (en segundos) usando el teclado.
Si presiona '*', inicia el proceso de aceleración.


\item Aceleración del motor:

El motor aumenta su velocidad gradualmente desde 0 hasta el máximo durante el tiempo ingresado.
En cada paso, actualiza la potencia del motor (con PWM) y calcula las RPM usando un sensor y una rutina de interrupción.
Muestra el porcentaje de aceleración y las RPM en la pantalla LCD.

\item Velocidad máxima:

Mantiene el motor a velocidad máxima por 5 segundos, mostrando las RPM.

\item Finalización:

Apaga el motor, muestra un mensaje de finalización y vuelve a la pantalla inicial.

\end{itemize}

\subsection{ Código C}

\lstinputlisting[language=C]{p1_Embebidos2/p1_Embebidos2.ino}


\section{Observaciones y Conclusiones}

\begin{itemize}
    \item Nuestro motor al ser más de potenica no tiene tantas RPM, lo cuál creo facilito la práctica.
    \item Está complicado mantener que el motor no se pare con el sensor que mide RPM.
    \item Está practica a sido de las más complicadas, ya que implica unir el puente H, el motor, y el sensor que mide RPM, todo esto además de un algoritmo algo más 
    complejo debido al cambio gradual de velocidad, pero al ya haber usado el motor y puente H en nuestro proyecto de Embebidos 1 se nos facilito bastante. 
\end{itemize}
  
%\nocite{*}

%\printbibliography[
%heading=bibintoc,
%title={ . }
%]
    \end{document}